\section{Agent Methodology}
We apply the GAIA methodology to the cooperative Multi-Agent System. It allows us to get on organizational view of the system being modeled. We let the naive Multi-Agent System aside as there are no interactions between agents.

\subsection{Role model}
This Agent-based System contains only one role : the Philosopher.
\begin{itemize}
\item{Role schema: }Philosopher
\item{Description: }This role involves thinking as much as possible and helping its neighbors to eat if possible by releasing a fork if it allows the neighbor to eat.
\item{Protocols and Activities: }Think, Starve, Eat, TakeLeftFork, TakeRightFork, ReleaseLeftFork, ReleaseRightFork, TellNeighbors.
\item{Permissions: }
    \begin{itemize}
    \item reads leftForkStatus // Is the left fork available or wanted by another Philosopher
    \item reads rightForkStatus // Is the right for available or wanted by another Philosopher
    \end{itemize}
    \begin{itemize}
    \item changes leftForkMarked // Mark the left fork as wanted
    \item changes rightForkMarked // Mark the right fork as wanted
    \item changes leftForkTaken // Take or release the left fork
    \item  changesrightForkTaken // Take or release the right fork
    \end{itemize}

\item{Liveness: }
                PHILOSOPHER: ( Think, Starve, Take, Eat, Release $)^\omega$\\
                TAKE: (TakeLeftFork, TakeRightFork)\\
                RELEASE: (ReleaseLeftFork, ReleaseRightFork, TellNeighbors)\\
\item{Safety: }
    \begin{itemize}
        \item true
    \end{itemize}
\end{itemize}

\subsection{Interaction model}
There are two protocols in the interaction model : \textit{Take} and \textit{Release}\\
\\
\textit{Take}
\begin{itemize}
\item{Purpose: }Take a fork
\item{Initiator: }Philosopher
\item{Responder: }Philosopher
\item{Inputs: }state = \text{STARVING}
\item{Outputs: }Fork Status
\item{Processing: }The initiator is hungry and ask his neighbors if their fork are available for taking, if yes take them, otherwise wait and try again.
\end{itemize}
\textit{Release}
\begin{itemize}
\item{Purpose: }Take a fork
\item{Initiator: }Philosopher
\item{Responder: }---
\item{Inputs: }---
\item{Outputs: }---
\item{Processing: }When the initiator is done eating, release the forks and notify the neighbors.
\end{itemize}

\subsection{Behavior rules}
\begin{itemize}
\item If an agent is starving, try to take the left fork and the right fork.
\item If an agent holds only one fork which is aready marked, release it.
\item When an agent is done eating, he should release immediately his forks.
\end{itemize}

\subsection{Agent model}
\begin{itemize}
\item A Philosopher is responsible for his state as well as his neighbors's.
\item A Philosopher divides the shared resources between himself and his neighbors, he always seeks to do the best use of the shared resources, even if it means waiting before taking a resource.
\item A Philosopher must update the state of the resources accordingly to his objective.
\end{itemize}
