This kind is self fish and will act to satisfy his needs.\\
Basically, the goal is to eat when the philosopher is hungry. Then when the philosopher become hungry he will just try to get the 2 forks to eat and if he has the 2 forks, he starts to eat until he becomes full.\\
This strategy is really simple because the agent (or the philosopher) just look at the forks, if they are available or not and then take it to eat wihtout consider other agent needs.\\
So for this kind of philosopher it's a reactive agent that is used because this behaviour is just a series of if. So the agent just react to different situation, there is no deliberation to reach his goal.