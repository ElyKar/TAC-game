An architecture for an agent define the global work methods of the agent.\\
It exists many architectures but 2 are more famous:\\
\begin{itemize}
	\item Reactive Agent
	\item Reasoning Agent
\end{itemize}\\
The first one is when there are many behaviours developped for and agent and then when an event occurs, the agent react to this event by chosing one of his behaviour. So the agent just react when events occurs.\\
Whereas the reasoning agent will look at his environement and his goal then he will deliberate to find a solution to reach his goal regarding his starting state.\\
The second one is more intelligent but it's not very effective for simple tasks which need quick reaction.\\
These 2 architectures are different but sometimes you can mix both to create an hybrid agent. So you can adapt to many situations.\\
In this part we will describe the architecture for the 2 kind of agent in the philosophy dinner problem, for the naive agent (whithout cooperation) and for the smart agent(with cooperation).

