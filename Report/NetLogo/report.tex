
%% bare_conf.tex
%% V1.4b
%% 2015/08/26
%% by Michael Shell
%% See:
%% http://www.michaelshell.org/
%% for current contact information.
%%
%% This is a skeleton file demonstrating the use of IEEEtran.cls
%% (requires IEEEtran.cls version 1.8b or later) with an IEEE
%% conference paper.
%%
%% Support sites:
%% http://www.michaelshell.org/tex/ieeetran/
%% http://www.ctan.org/pkg/ieeetran
%% and
%% http://www.ieee.org/

%%*************************************************************************
%% Legal Notice:
%% This code is offered as-is without any warranty either expressed or
%% implied; without even the implied warranty of MERCHANTABILITY or
%% FITNESS FOR A PARTICULAR PURPOSE! 
%% User assumes all risk.
%% In no event shall the IEEE or any contributor to this code be liable for
%% any damages or losses, including, but not limited to, incidental,
%% consequential, or any other damages, resulting from the use or misuse
%% of any information contained here.
%%
%% All comments are the opinions of their respective authors and are not
%% necessarily endorsed by the IEEE.
%%
%% This work is distributed under the LaTeX Project Public License (LPPL)
%% ( http://www.latex-project.org/ ) version 1.3, and may be freely used,
%% distributed and modified. A copy of the LPPL, version 1.3, is included
%% in the base LaTeX documentation of all distributions of LaTeX released
%% 2003/12/01 or later.
%% Retain all contribution notices and credits.
%% ** Modified files should be clearly indicated as such, including  **
%% ** renaming them and changing author support contact information. **
%%*************************************************************************


% *** Authors should verify (and, if needed, correct) their LaTeX system  ***
% *** with the testflow diagnostic prior to trusting their LaTeX platform ***
% *** with production work. The IEEE's font choices and paper sizes can   ***
% *** trigger bugs that do not appear when using other class files.       ***                          ***
% The testflow support page is at:
% http://www.michaelshell.org/tex/testflow/



\documentclass[conference]{IEEEtran}
% Some Computer Society conferences also require the compsoc mode option,
% but others use the standard conference format.
%
% If IEEEtran.cls has not been installed into the LaTeX system files,
% manually specify the path to it like:
% \documentclass[conference]{../sty/IEEEtran}





% Some very useful LaTeX packages include:
% (uncomment the ones you want to load)


% *** MISC UTILITY PACKAGES ***
%
%\usepackage{ifpdf}
% Heiko Oberdiek's ifpdf.sty is very useful if you need conditional
% compilation based on whether the output is pdf or dvi.
% usage:
% \ifpdf
%   % pdf code
% \else
%   % dvi code
% \fi
% The latest version of ifpdf.sty can be obtained from:
% http://www.ctan.org/pkg/ifpdf
% Also, note that IEEEtran.cls V1.7 and later provides a builtin
% \ifCLASSINFOpdf conditional that works the same way.
% When switching from latex to pdflatex and vice-versa, the compiler may
% have to be run twice to clear warning/error messages.






% *** CITATION PACKAGES ***
%
%\usepackage{cite}
% cite.sty was written by Donald Arseneau
% V1.6 and later of IEEEtran pre-defines the format of the cite.sty package
% \cite{} output to follow that of the IEEE. Loading the cite package will
% result in citation numbers being automatically sorted and properly
% "compressed/ranged". e.g., [1], [9], [2], [7], [5], [6] without using
% cite.sty will become [1], [2], [5]--[7], [9] using cite.sty. cite.sty's
% \cite will automatically add leading space, if needed. Use cite.sty's
% noadjust option (cite.sty V3.8 and later) if you want to turn this off
% such as if a citation ever needs to be enclosed in parenthesis.
% cite.sty is already installed on most LaTeX systems. Be sure and use
% version 5.0 (2009-03-20) and later if using hyperref.sty.
% The latest version can be obtained at:
% http://www.ctan.org/pkg/cite
% The documentation is contained in the cite.sty file itself.






% *** GRAPHICS RELATED PACKAGES ***
%
\ifCLASSINFOpdf
% \usepackage[pdftex]{graphicx}
  % declare the path(s) where your graphic files are
  % \graphicspath{{../pdf/}{../jpeg/}}
  % and their extensions so you won't have to specify these with
  % every instance of \includegraphics
  % \DeclareGraphicsExtensions{.pdf,.jpeg,.png}
\else
  % or other class option (dvipsone, dvipdf, if not using dvips). graphicx
  % will default to the driver specified in the system graphics.cfg if no
  % driver is specified.
  % \usepackage[dvips]{graphicx}
  % declare the path(s) where your graphic files are
  % \graphicspath{{../eps/}}
  % and their extensions so you won't have to specify these with
  % every instance of \includegraphics
  % \DeclareGraphicsExtensions{.eps}
\fi
% graphicx was written by David Carlisle and Sebastian Rahtz. It is
% required if you want graphics, photos, etc. graphicx.sty is already
% installed on most LaTeX systems. The latest version and documentation
% can be obtained at: 
% http://www.ctan.org/pkg/graphicx
% Another good source of documentation is "Using Imported Graphics in
% LaTeX2e" by Keith Reckdahl which can be found at:
% http://www.ctan.org/pkg/epslatex
%
% latex, and pdflatex in dvi mode, support graphics in encapsulated
% postscript (.eps) format. pdflatex in pdf mode supports graphics
% in .pdf, .jpeg, .png and .mps (metapost) formats. Users should ensure
% that all non-photo figures use a vector format (.eps, .pdf, .mps) and
% not a bitmapped formats (.jpeg, .png). The IEEE frowns on bitmapped formats
% which can result in "jaggedy"/blurry rendering of lines and letters as
% well as large increases in file sizes.
%
% You can find documentation about the pdfTeX application at:
% http://www.tug.org/applications/pdftex





% *** MATH PACKAGES ***
%
%\usepackage{amsmath}
% A popular package from the American Mathematical Society that provides
% many useful and powerful commands for dealing with mathematics.
%
% Note that the amsmath package sets \interdisplaylinepenalty to 10000
% thus preventing page breaks from occurring within multiline equations. Use:
%\interdisplaylinepenalty=2500
% after loading amsmath to restore such page breaks as IEEEtran.cls normally
% does. amsmath.sty is already installed on most LaTeX systems. The latest
% version and documentation can be obtained at:
% http://www.ctan.org/pkg/amsmath





% *** SPECIALIZED LIST PACKAGES ***
%
%\usepackage{algorithmic}
% algorithmic.sty was written by Peter Williams and Rogerio Brito.
% This package provides an algorithmic environment fo describing algorithms.
% You can use the algorithmic environment in-text or within a figure
% environment to provide for a floating algorithm. Do NOT use the algorithm
% floating environment provided by algorithm.sty (by the same authors) or
% algorithm2e.sty (by Christophe Fiorio) as the IEEE does not use dedicated
% algorithm float types and packages that provide these will not provide
% correct IEEE style captions. The latest version and documentation of
% algorithmic.sty can be obtained at:
% http://www.ctan.org/pkg/algorithms
% Also of interest may be the (relatively newer and more customizable)
% algorithmicx.sty package by Szasz Janos:
% http://www.ctan.org/pkg/algorithmicx




% *** ALIGNMENT PACKAGES ***
%
%\usepackage{array}
% Frank Mittelbach's and David Carlisle's array.sty patches and improves
% the standard LaTeX2e array and tabular environments to provide better
% appearance and additional user controls. As the default LaTeX2e table
% generation code is lacking to the point of almost being broken with
% respect to the quality of the end results, all users are strongly
% advised to use an enhanced (at the very least that provided by array.sty)
% set of table tools. array.sty is already installed on most systems. The
% latest version and documentation can be obtained at:
% http://www.ctan.org/pkg/array


% IEEEtran contains the IEEEeqnarray family of commands that can be used to
% generate multiline equations as well as matrices, tables, etc., of high
% quality.




% *** SUBFIGURE PACKAGES ***
\usepackage{verbatimbox}
\ifCLASSOPTIONcompsoc
  \usepackage[caption=false,font=normalsize,labelfont=sf,textfont=sf]{subfig}
\else
  \usepackage[caption=false,font=footnotesize]{subfig}
\fi
% subfig.sty, written by Steven Douglas Cochran, is the modern replacement
% for subfigure.sty, the latter of which is no longer maintained and is
% incompatible with some LaTeX packages including fixltx2e. However,
% subfig.sty requires and automatically loads Axel Sommerfeldt's caption.sty
% which will override IEEEtran.cls' handling of captions and this will result
% in non-IEEE style figure/table captions. To prevent this problem, be sure
% and invoke subfig.sty's "caption=false" package option (available since
% subfig.sty version 1.3, 2005/06/28) as this is will preserve IEEEtran.cls
% handling of captions.
% Note that the Computer Society format requires a larger sans serif font
% than the serif footnote size font used in traditional IEEE formatting
% and thus the need to invoke different subfig.sty package options depending
% on whether compsoc mode has been enabled.
%
% The latest version and documentation of subfig.sty can be obtained at:
% http://www.ctan.org/pkg/subfig




% *** FLOAT PACKAGES ***
%
%\usepackage{fixltx2e}
% fixltx2e, the successor to the earlier fix2col.sty, was written by
% Frank Mittelbach and David Carlisle. This package corrects a few problems
% in the LaTeX2e kernel, the most notable of which is that in current
% LaTeX2e releases, the ordering of single and double column floats is not
% guaranteed to be preserved. Thus, an unpatched LaTeX2e can allow a
% single column figure to be placed prior to an earlier double column
% figure.
% Be aware that LaTeX2e kernels dated 2015 and later have fixltx2e.sty's
% corrections already built into the system in which case a warning will
% be issued if an attempt is made to load fixltx2e.sty as it is no longer
% needed.
% The latest version and documentation can be found at:
% http://www.ctan.org/pkg/fixltx2e


%\usepackage{stfloats}
% stfloats.sty was written by Sigitas Tolusis. This package gives LaTeX2e
% the ability to do double column floats at the bottom of the page as well
% as the top. (e.g., "\begin{figure*}[!b]" is not normally possible in
% LaTeX2e). It also provides a command:
%\fnbelowfloat
% to enable the placement of footnotes below bottom floats (the standard
% LaTeX2e kernel puts them above bottom floats). This is an invasive package
% which rewrites many portions of the LaTeX2e float routines. It may not work
% with other packages that modify the LaTeX2e float routines. The latest
% version and documentation can be obtained at:
% http://www.ctan.org/pkg/stfloats
% Do not use the stfloats baselinefloat ability as the IEEE does not allow
% \baselineskip to stretch. Authors submitting work to the IEEE should note
% that the IEEE rarely uses double column equations and that authors should try
% to avoid such use. Do not be tempted to use the cuted.sty or midfloat.sty
% packages (also by Sigitas Tolusis) as the IEEE does not format its papers in
% such ways.
% Do not attempt to use stfloats with fixltx2e as they are incompatible.
% Instead, use Morten Hogholm'a dblfloatfix which combines the features
% of both fixltx2e and stfloats:
%
% \usepackage{dblfloatfix}
% The latest version can be found at:
% http://www.ctan.org/pkg/dblfloatfix




% *** PDF, URL AND HYPERLINK PACKAGES ***
%
%\usepackage{url}
% url.sty was written by Donald Arseneau. It provides better support for
% handling and breaking URLs. url.sty is already installed on most LaTeX
% systems. The latest version and documentation can be obtained at:
% http://www.ctan.org/pkg/url
% Basically, \url{my_url_here}.

\usepackage[T1]{fontenc}
\usepackage[utf8]{inputenc}
\usepackage{color}
\usepackage[english]{babel}
\usepackage{graphicx}
\usepackage{makecell}
\usepackage{enumitem}
\usepackage{hyperref}
\usepackage{float}
\usepackage{mathtools}



% *** Do not adjust lengths that control margins, column widths, etc. ***
% *** Do not use packages that alter fonts (such as pslatex).         ***
% There should be no need to do such things with IEEEtran.cls V1.6 and later.
% (Unless specifically asked to do so by the journal or conference you plan
% to submit to, of course. )


% correct bad hyphenation here
\hyphenation{op-tical net-works semi-conduc-tor}


\begin{document}
%
% paper title
% Titles are generally capitalized except for words such as a, an, and, as,
% at, but, by, for, in, nor, of, on, or, the, to and up, which are usually
% not capitalized unless they are the first or last word of the title.
% Linebreaks \\ can be used within to get better formatting as desired.
% Do not put math or special symbols in the title.
\title{NetLogo - Dining Philosophers}


% author names and affiliations
% use a multiple column layout for up to three different
% affiliations
\author{\IEEEauthorblockN{Tristan Claverie}
\IEEEauthorblockA{950418P612 \\ trcl16@student.bth.se}
\and
\IEEEauthorblockN{Damien Duvacher}
\IEEEauthorblockA{941206P633 \\ dadu16@student.bth.se}}

% conference papers do not typically use \thanks and this command
% is locked out in conference mode. If really needed, such as for
% the acknowledgment of grants, issue a \IEEEoverridecommandlockouts
% after \documentclass

% for over three affiliations, or if they all won't fit within the width
% of the page, use this alternative format:
% 
%\author{\IEEEauthorblockN{Michael Shell\IEEEauthorrefmark{1},
%Homer Simpson\IEEEauthorrefmark{2},
%James Kirk\IEEEauthorrefmark{3}, 
%Montgomery Scott\IEEEauthorrefmark{3} and
%Eldon Tyrell\IEEEauthorrefmark{4}}
%\IEEEauthorblockA{\IEEEauthorrefmark{1}School of Electrical and Computer Engineering\\
%Georgia Institute of Technology,
%Atlanta, Georgia 30332--0250\\ Email: see http://www.michaelshell.org/contact.html}
%\IEEEauthorblockA{\IEEEauthorrefmark{2}Twentieth Century Fox, Springfield, USA\\
%Email: homer@thesimpsons.com}
%\IEEEauthorblockA{\IEEEauthorrefmark{3}Starfleet Academy, San Francisco, California 96678-2391\\
%Telephone: (800) 555--1212, Fax: (888) 555--1212}
%\IEEEauthorblockA{\IEEEauthorrefmark{4}Tyrell Inc., 123 Replicant Street, Los Angeles, California 90210--4321}}




% use for special paper notices
%\IEEEspecialpapernotice{(Invited Paper)}




% make the title area
\maketitle

\begin{abstract}
The dining philosophers is a very common synchronization problem. It is often used as a learning material to teach students synchronization issues such as deadlocks and techniques to solve them. There have been many variants and solutions of this problem. The purpose of the studied program is to solve this problem using a Multi-Agent System
\end{abstract}
% As a general rule, do not put math, special symbols or citations
% in the abstract

% no keywords




% For peer review papers, you can put extra information on the cover
% page as needed:
% \ifCLASSOPTIONpeerreview
% \begin{center} \bfseries EDICS Category: 3-BBND \end{center}
% \fi
%
% For peerreview papers, this IEEEtran command inserts a page break and
% creates the second title. It will be ignored for other modes.
\IEEEpeerreviewmaketitle




In computer science, an agent is a software who works autonomous. It acts like an automaton according to how it was developped.\\
 It's a kind of artificial intelligence.\\
  An agent can be :\\
\begin{itemize}
	\item Reactive: It can act regarding to his environment.
	\item Proactive: It can take initiative to reach its goal.
	\item "Sociable": It can communicate with other agents.
\end{itemize}
Agent can be use in many fields, for example it there is an agent to organize import and export of containers in a container terminal at Göteborg,this agent permits to fill, empty and stack containers from ships in the dock via machines. Machine moving are managed by agents and are optimized to be the quickest as possible.\\
An other example very much simple is just a reactive agent which can sent automatic responses when you order a command or something else.\\
So, agent are usefull to do some tasks and can make the life easier.\\
The TAC agent competition is an international forum designed to promote and encourage high quality research into the trading agent problem.\\
In the TAC classic, we are an agent reponsible of 8 clients who want to go in holliday. We have to create travel packages to answer to their preferences (Flight date, Hotel preferences, Entertainements asked).\\
We have to participate to different auctions to make the packages. There are 3 different auctions:\\
\begin{itemize}
\item Flight auctions
\item Hotel auctions
\item Entertainement auctions
\end{itemize}
We will describe each of behaviour auction in the rest of this report.\\
We will participate to a competition between each agents of the different group of student.\\
At the end of the competion we will have a score regarding our holding and the higher score will win.\\
So we have to design an agent to win this competition, this report will describe how we've developped our agent.

\section{Agent Architecture}

An architecture for an agent define the global work methods of the agent.\\
It exists many architectures but 2 are more famous:\\
\begin{itemize}
	\item Reactive Agent
	\item Reasoning Agent
\end{itemize}
The first one is when there are many behaviours developped for and agent and then when an event occurs, the agent react to this event by chosing one of his behaviour. So the agent just react when events occurs.\\
Whereas the reasoning agent will look at his environement and his goal then he will deliberate to find a solution to reach his goal regarding his starting state.\\
The second one is more intelligent but it's not very effective for simple tasks which need quick reaction.\\
These 2 architectures are different but sometimes you can mix both to create an hybrid agent. So you can adapt to many situations.\\
In this part we will describe the architecture for the 2 kind of agent in the philosophy dinner problem, for the naive agent (whithout cooperation) and for the smart agent(with cooperation).

\subsection{Naive Agent}
This kind is self fish and will act to satisfy his needs.\\
Basically, the goal is to eat when the philosopher is hungry. Then when the philosopher become hungry he will just try to get the 2 forks to eat and if he has the 2 forks, he starts to eat until he becomes full.\\
This strategy is really simple because the agent (or the philosopher) just look at the forks, if they are available or not and then take it to eat wihtout consider other agent needs.\\
So for this kind of philosopher it's a reactive agent that is used because this behaviour is just a series of if. So the agent just react to different situation, there is no deliberation to reach his goal.

\subsection{Smart Agent}
This philosopher is a little bit more smart because he will look at the marks on the forks and will mark the forks that he need. The goal is the same: eat when hungry. But this timethe agent be careful of other agents needs.\\
Despite this difference the architecture for this smart philospoher is the same than the naive: a reactive agent.\\
Because the difference is about the marks on the forks, and the marks are managed by a "if series" too, indeed, if a philospoh has a fork marked and he doesn't have the other fork he will put down the fork because the mark mean that an other philosoph need it.\\
As you can see before it's just manage by many "if", he doesn't think about his environment and how he can reach his goal.


\section{Agent Methodology}
We apply the GAIA methodology to the cooperative Multi-Agent System. It allows us to get on organizational view of the system being modeled. We let the naive Multi-Agent System aside as there are no interactions between agents.

\subsection{Role model}
This Agent-based System contains only one role : the Philosopher.
\begin{itemize}
\item{Role schema: }Philosopher
\item{Description: }This role involves thinking as much as possible and helping its neighbors to eat if possible by releasing a fork if it allows the neighbor to eat.
\item{Protocols and Activities: }Think, Starve, Eat, TakeLeftFork, TakeRightFork, ReleaseLeftFork, ReleaseRightFork, TellNeighbors.
\item{Permissions: }
    \begin{itemize}
    \item reads leftForkStatus // Is the left fork available or wanted by another Philosopher
    \item reads rightForkStatus // Is the right for available or wanted by another Philosopher
    \end{itemize}
    \begin{itemize}
    \item changes leftForkMarked // Mark the left fork as wanted
    \item changes rightForkMarked // Mark the right fork as wanted
    \item changes leftForkTaken // Take or release the left fork
    \item  changesrightForkTaken // Take or release the right fork
    \end{itemize}

\item{Liveness: }
                PHILOSOPHER: ( Think, Starve, Take, Eat, Release $)^\omega$\\
                TAKE: (TakeLeftFork, TakeRightFork)\\
                RELEASE: (ReleaseLeftFork, ReleaseRightFork, TellNeighbors)\\
\item{Safety: }
    \begin{itemize}
        \item true
    \end{itemize}
\end{itemize}

\subsection{Interaction model}
There are two protocols in the interaction model : \textit{Take} and \textit{Release}\\
\\
\textit{Take}
\begin{itemize}
\item{Purpose: }Take a fork
\item{Initiator: }Philosopher
\item{Responder: }Philosopher
\item{Inputs: }state = \text{STARVING}
\item{Outputs: }Fork Status
\item{Processing: }The initiator is hungry and ask his neighbors if their fork are available for taking, if yes take them, otherwise wait and try again.
\end{itemize}
\textit{Release}
\begin{itemize}
\item{Purpose: }Take a fork
\item{Initiator: }Philosopher
\item{Responder: }---
\item{Inputs: }---
\item{Outputs: }---
\item{Processing: }When the initiator is done eating, release the forks and notify the neighbors.
\end{itemize}

\subsection{Behavior rules}
\begin{itemize}
\item If an agent is starving, try to take the left fork and the right fork.
\item If an agent holds only one fork which is aready marked, release it.
\item When an agent is done eating, he should release immediately his forks.
\end{itemize}

\subsection{Agent model}
\begin{itemize}
\item A Philosopher is responsible for his state as well as his neighbors's.
\item A Philosopher divides the shared resources between himself and his neighbors, he always seeks to do the best use of the shared resources, even if it means waiting before taking a resource.
\item A Philosopher must update the state of the resources accordingly to his objective.
\end{itemize}


\section{Agent Communication and Interaction}
In this part we will talk about the interaction and the communication between the different agents in this multi-agent system. We will first see the intercation between naive agents and then for the smart agents.

\subsection{Naive agent}
For the naive version of the agent, the communication is not really advanced because the philospohers don't communicate between them. One philospoh will just communicate with the fork, he will just look if they are available or not. And the interaction is not really complicated neither because a philosopher can just take or release forks. If we have to find a slight part of interaction between philosophers, it happens that when a philosopher have 2 forks another one can't have his 2 forks. However, it would be more accurate to call it a consequence of the system state given its constraints rather than a real interaction.\\
So for the naive version there is almost no interaction and no communication at all.

\subsection{Smart agent}
This kind of philosopher is more interesting because smart philosopher are supposed to cooperate to help each other to reach their goal: eat as soon as possible and think for as long as possible.\\
Here there is a communication between philosopher because they mark a fork, that means they want this fork and if a philosopher have a marked fork but doesn't have the other one he will release this fork because someone else needs it.\\
It's a kind of communication not directly but by the forks, they send informations to each other marking forks they need. Also, when releasing the forks they are notifying the other philosophers that they did.
The interaction between philosopher is the same that the naive agent.


\section*{Concusion}

In the end, we believe our ideas for the TAC game were decent and worked well, except for the hotel auction.\\
We chose to optimize client satisfaction, however another choice could have been to maximize the final score : difference between client satisfaction and insatisfaction.\\
Furthermore, we provide three other ways to improve our agent, but they are difficult to implement and we prefered to focus on having a less performant, working agent rather than a more performant, buggy agent.
\begin{itemize}
    \item Using a genetic algorithm, we could have had a dynamic optimal state of the game, refreshing at each update in the game (auction closed, bid accepted, auction updated, \dots ). This way, we could ensure an optimal or quasi-optimal set of packages for a reasonable price.
    \item With machine learning over the auctions of the hotels, we might have been able to infer which set of nights did the other agent wanted and go for the other hotel if the competition was too high.
    \item Using Markov chains, it would have been possible to predict the next state of the game according to the current one and react accordingly. However, this is hardest strategy to implement out of the three.
\end{itemize}

Finally, we were pleased to take part in this competition, even though our agent was lacking in some ways.



% trigger a \newpage just before the given reference
% number - used to balance the columns on the last page
% adjust value as needed - may need to be readjusted if
% the document is modified later
%\IEEEtriggeratref{8}
% The "triggered" command can be changed if desired:
%\IEEEtriggercmd{\enlargethispage{-5in}}

% references section

% can use a bibliography generated by BibTeX as a .bbl file
% BibTeX documentation can be easily obtained at:
% http://mirror.ctan.org/biblio/bibtex/contrib/doc/
% The IEEEtran BibTeX style support page is at:
% http://www.michaelshell.org/tex/ieeetran/bibtex/
%
% <OR> manually copy in the resultant .bbl file
% set second argument of \begin to the number of references
% (used to reserve space for the reference number labels box)

% that's all folks
\end{document}



